\documentclass{life-fr}

\begin{document}

\title{La Vie est un Jeu}
\subtitle{Étude de l'existant}
\member{Lepage Barbara}{db0company@gmail.com}
\member{Caradec Guillaume}{guillaume.caradec@gmail.com }
\member{El-Outmani Youssef}{youssef.eloutmani@gmail.com}
\member{Glorieux François}{fra.glorieux@gmail.com}
\member{Klarman Nicolas}{nickoas@gmail.com}
\member{Lassagne David}{david.lassagne@gmail.com}
\member{Le-Cor Wilfried}{wilfried.lecor@gmail.com}
\member{Lenormand Frank}{lenormf@gmail.com}
\member{Louvigny Guillaume}{guillaume@louvigny.fr}

\summary
{
  Ce document présente les projets existants ressemblant au notre. Il nous permettra de faire une étude du marché.
}

\maketitle

\tableofcontents

\chapter{Rappel de l’EIP}
\section{Qu’est-ce qu’un EIP et Epitech}
Epitech est une école formant des experts en informatique. Sa pédagogie par projet implique directement les étudiants dans leur apprentissage et les rend plus à même de réagir et s'adapter aisément, par exemple aux évolutions technologiques qui auront lieu au cours de leur carrière.\\

Un Epitech Innovative Project ou EIP est l'élément clé du cursus Epitech. Il s'agit d'un projet de fin d'études regroupant un minimum de six étudiants autour d'un but commun. Ce projet est conduit sur une durée de trois ans, beaucoup plus importante que celles des projets réalisés lors des trois premières années d'études. De plus, l’EIP amène les étudiants à se confronter au monde de l’entreprise.

\section{ujet de votre EIP}
Dans le cadre de notre EIP, nous avons décidé de réaliser un réseau social à but ludique basé sur les « bucket lists ». Une bucket list est « une liste de chose à faire avant de mourir » : elle définit toutes les choses que son auteur désire faire de son existence, une sorte de mémo pour ne pas gâcher sa vie. Notre projet permettra à nos visiteurs de construire leurs propres listes, faire valider leurs exploits, tout en le partageant avec leurs réseaux sociaux. Ainsi, chaque action réalisée par un utilisateur (ajout d’une activité, succès ou échec) sera un fil de discussion dans lequel le visiteur et son réseau pourra discuter et partager différents types de médias (photos, vidéos, etc). L’activité d’un utilisateur sera validée par son propre réseau et apparaîtra sous forme de “succès”, comme dans un jeu vidéo. Le site s’étendra par la suite en proposant d’autres caractéristiques propres aux jeux vidéos.

\chapter{Les projets existants}

\section{EpicWin}
\url{http://www.rexbox.co.uk/epicwin/}\\

\begin{itemize}

  \item EpicWin est une application uniquement disponible sur iOS.
  \item Elle est dotée d'une forte inspiration Viking dans son thème graphique.
  \item De part sa nature (une application) elle est peu ouverte.
  \item On ne trouve pas d'API pour accéder aux données, personne ne peut donc y ajouter des fonctionnalités.
  \item Les actions à réaliser pour débloquer des objectifs sont principalement ceux de la vie quotidienne (faire sa vaisselle, aller en salle de sport etc.), ne laissant pas une réelle place à ce que les utilisateurs serait fiers de montrer à leurs amis.
  \item On notera également que sa dernière mise à jour date de l'année dernière.
  \item Ainsi que sa non disponibilité pour les terminaux Android ou Windows Phone, ou encore depuis un navigateur internet (mobile, PC).
\end{itemize}

\section{Bucketlist.org}
\url{http://bucketlist.org/}

\begin{itemize}
  \item Bucketlist.org est un site permettant de publier sa bucketlist : ce qui a été déjà fait, ce qu'il reste à faire.
  \item Les objectifs sont proposés par les utilisateurs, et tout le monde est libre de se joindre à un objectif soumis par quelqu'un d'autre.
  \item La dimension sociale est peu mise en avant, seule la présence du site (via sa page ``fan'') sur les réseaux sociaux ainsi que des boutons de type ``Like'' sont visibles de prime abord.
\end{itemize}

\section{Bucketlist.net}
\url{http://bucketlist.net/}

\begin{itemize}
  \item Bucketlist.net est un site de partage, et non un réseau social.
  \item Le principe étant de rédiger sa bucket list sur le site, d’aider les autres à réaliser leurs objectifs et de bénéficier de l’aide des autres en retour.
  \item La dimension ludique est inéxistante et le partage d’expériences est peu mise en avant.
\end{itemize}

\section{OneFeat}
\url{http://onefeat.com/}

\begin{itemize}
  \item Onefeat.com est un site permettant d’accomplir des missions par le biais d’upload d’images.
  \item Il est très ludique mais ne permet de construire une réelle bucket list.
  \item Ainsi, la plateforme reste un jeu avant tout : il faut upload des photos de plus en plus difficile à obtenir, et gagner de plus en plus de points.
  \item Le fait de montrer une photo est plus mise en avant que de réaliser l’objectif associé.
  \item De plus, la dimension sociale reste limitée...
\end{itemize}

\chapter{Positionnement de votre projet}

\section{Ce que vous apportez}

Parmi tout les projets existants, aucun ne combine les éléments que nous voulons mettre en avant. En effet, nous tenons à mettre en place une plate-forme suffisamment ludique pour être visitée au quotidien, sans pour autant négliger l’aspect social permettant au visiteur de discuter de ses loisirs avec son réseau et de pouvoir partager aisément ses expériences à travers photos et vidéos.

\section{Ce qui ne sera pas couvert}

Le projet étant avant tout un réseau social, la dimension sociale sera nécessairement limitée, et non pas omniprésente comme le epicwin, un des projets existants présenté plus haut. En outre, le système d’entraide sur lequel se base bucketlist.net (un autre projet présenté plus haut) ne sera pas abordé, le principe étant différent de ce que nous voulons faire.\\

Cependant, nous envisageons de proposer aux utilisateurs des achievements à réaliser en groupe.

\chapter{Conclusion}

Pour conclure, notre projet est un mix de plusieurs projets déjà disponible en une seule plateforme. Cette plateforme est de type réseau social, ce qui plait le plus aux personnes de nos jours.
Notre plateforme donne aux personnes la chance de transformer leurs vies quotidienne en une vie incroyable pleine d'aventures !

\end{document}
