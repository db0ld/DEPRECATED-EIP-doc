\documentclass{life-fr}

\begin{document}

\title{Démo LaTeX}
\subtitle{DocumentClass Life v2}
\member{Barbara Lepage}{db0company@gmail.com}

\summary
{
  Ce document montre les différentes fonctionnalités de la
  documentclass EIP.
}

\maketitle

\tableofcontents

 \chapter{Exemple de mise en forme}

   \section{Lien}
     Ceci est un lien :
     \url{http://google.com}\\

     Vous conaissez Google ?

   \section{Mise en forme du texte}
     \textbf{texte en gras}\\
     \textit{texte en italique}\\
     \underline{texte souligne}\\
     texte normal

   \section{Insertion d'une image}
     \begin{figure}[H]
      \begin{center}
        \includegraphics[width=15cm]{../../../logo/lveuj_cuteface.png}
      \end{center}
    \end{figure} 

 \chapter{D'autres exemples}
 
   \section{De jolies boîtes}
     Pour mettre en avant du texte :
     \warn {Vous devez faire attention à ce message.}
     \info {Ce message apporte une information.}
     \hint {Ceci est un conseil à suivre.}
     \yes{C'est terminé, bouclé, validé !}
     \ongoing{C'est en cours...}
     \nobox{Ce n'est pas fait, c'est faux, c'est une négation.}
     \newpage 

   \section{Une liste}
     Et pour finir, une petite liste :
     \begin{itemize}
      \item élement
      \item autre élement
      \item Pourquoi pas une sous-liste ?
        \begin{itemize}
          \item hello
          \item world
        \end{itemize}
      \item Et encore un élément !
     \end{itemize}

\chapter{Extrait d'un cours réel}

    Vous devez utiliser l'interprete pour resoudre les exercices de ce TP. Je
  vous rappel que la commande shell pour lancer l'interprete OCaml est
  \texttt{"ocaml"}.\\
\\  Exemple :

  \begin{lstlisting}
    >ocaml
             Objective Caml version 3.10.2

    #
  \end{lstlisting}

  \hint
  {
    Vous pouvez utiliser la commande rlwrap avec en parametre l'interprete ocaml
    pour beneficier d'une edition de ligne confortable.
  }

    La version d'ocaml que vous utilisez peut varier de celle de l'exemple, mais
  cela n'influra pas sur le contenu de ce module.\\
  Pour quitter l'interprete, vous pouvez taper la directive \texttt{"\#quit"}
  (avec le \texttt{\#}) ou faire un \texttt{<ctrl> + d}.\\

  Si vous voulez conserver le code que vous allez taper dans ce TP, vous pouvez
  l'ecrire dans un fichier et le charger dans l'interprete grace a la directive
  \texttt{"\#use "fichier.ml""} (avec le \texttt{\#} et les doubles quotes autour
  du nom du fichier de sources). Cette directive va lire, compiler et evaluer
  votre fichier directement dans l'interprete.\\

  \hint
  {
    Le symbole ";;" sert a marquer la fin d'une commande dans l'interprete
    ocaml. Il est donc inutile de le mettre a la fin de vos expressions dans
    votre fichier de sources.
  }

  Exemple :

  \begin{lstlisting}
    >cat exemple.ml
    let greetings = "Bonjour les tech2 !" (* Notez l'absence de ";;" *)
    >ocaml
            Objective Caml version 3.11.1

    # #use "exemple.ml";;
    val greetings : string = "Bonjour les tech2 !"
    # greetings;;
    - : string = "Bonjour les tech2 !"
    #
  \end{lstlisting}


  \section{Exercice 1}

  \begin{itemize}
    \item Creez un fichiez nomme \texttt{"exercice\_1.ml"}
    \item Ecrivez dans ce fichier \texttt{"let exercice\_1 = "Reussi !""} suivi
      d'un retour a la ligne
    \item Sauvegardez, et dans votre shell, lancez l'interprete ocaml
    \item A l'invite de l'interprete, utilisez la directive \texttt{"\#use"}
      pour evaluer le fichier \texttt{"exercice\_1.ml"}
  \end{itemize}

\chapter{Expressions et types}

    Pour vous aider, je vous encourage tres fortement a consulter la
  documentation officielle du langage a l'adresse
  \url{http://caml.inria.fr/pub/docs/manual-ocaml/index.html} et la
  documentation du module Pervasives. Disponible a l'adresse
  \url{http://caml.inria.fr/pub/docs/manual-ocaml/libref/Pervasives.html}, ce
  module de la bibliotheque standard est ouvert par defaut dans tous vos
  programmes.


  \section{Exercice 2}

    Parcourez la documentation du module Pervasives a l'adresse
  \url{http://caml.inria.fr/pub/docs/manual-ocaml/libref/Pervasives.html},
  principalement la section "comparaisons".

  \section{Exercice 3}

    Tentez de predire le type et la valeur de chacune des expressions suivantes,
  puis verifiez si vous avez raison a l'aide de l'interprete. Il est tout a fait
  possible que certaines de ces expressions s'evaluent en erreurs...

  \begin{itemize}
    \item let a = 42;;
    \item a;;
    \item let b = "suspens...";;
    \item let c = ();;
    \item let d = 42 + 0;;
    \item let e = 42.0 +. 0;;
    \item let f = 30 and g = 12;;
    \item let h = f + g;;
    \item let i = let j = 50 and k = 8 in j - k;;
    \item let l = 42 in let m = l - 42 in l + m;;
    \item let n = 42 and o = n - 42 in n + o;;
  \end{itemize}


  \section{Exercice 4}

    Tentez de predire le type et la valeur de chacune des expressions suivantes,
  puis verifiez si vous avez raison a l'aide de l'interprete. Il est tout a fait
  possible que certaines de ces expressions s'evaluent en erreurs...

  \begin{itemize}
    \item let fonction\_p = fun a b -> a + b;;
    \item let fonction\_q a b = a + b;;
    \item fonction\_q 21 21;;
    \item fonction\_q;;
    \item let fonction\_r () = 42;;
    \item let fonction\_s a = 42;;
    \item let fonction\_t a = a;;
    \item let fonction\_u a b = a b;;
    \item let fonction\_v a b c = a b c;;
    \item let fonction\_w a b c = a (b c);;
    \item let fonction\_x () = let a = 42 in let b = 42 in a - b + 42;;
    \item let y = "a" in let fonction\_z a b = b $\wedge$ y in fonction\_z;;
    \item fonction\_z;;
  \end{itemize}

  \section{Exercice 5}

    Tentez de predire le type et la valeur de chacune des expressions suivantes,
  puis verifiez si vous avez raison a l'aide de l'interprete. Il est tout a fait
  possible que certaines de ces expressions s'evaluent en erreurs...

  \begin{itemize}
    \item let a = 42 in if a > 0 then true else false;;
    \item let str = "ocaml" in if str <> "" then print\_endline str;;
    \item if 42 = 24 then ( * ) else ( + );;
    \item let \_ = match 42 with 0 -> "zero" | n -> "42";;
    \item let rec f x = if x > 0 then (g (x - 1)) else 1 and g x = if x > 0 then
      (f (x - 1)) else 0;;
  \end{itemize}

\end{document}
